The introduction will be brief, and will introduce the reader to the document. It will outline the flow of the thesis: start with a description of neutrinos and some basic neutrino theory, then describe the MicroBooNE detector, then describe the BNB, then dive in to the Low Energy Excess section which contains description of LSND, description of MiniBooNE, description of the analysis in MicroBooNE, the sensitivity results in MicrobooNE. That section ends with describing how the flux systematic on intrinsuc nues from kaon decay (an important background in the LEE search) can be improved with an in-situ measurement of numus from kaons. Then comes the numu-from-kaon section, which references the importance of multiple Coulomb scattering and I will copy/paste the JINST publication in there. Then there will be a short conclusions/summary section, followed by any appendices which are needed.\\

``You have 3 topics in your thesis: LE, K’s and MCS. But they are very well related. SO a sentence or two in the Intro linking them would be great. Something along these lines:\\

In LE the intrinsic nue’s are an important irreducible background, of which a significant component is from kaons produced at the proton target decaying to nue’s. A measure of these kaons can be obtained from the studu of the high energy numuCC energy distribution. The momentim of the high energy muons produced in these events can only be measured by the multiple scatterin technique as they exit the cryostat precluding their measurement through range.'' --Leslie