This thesis describes work towards the search for a low energy excess in MicroBooNE. What MicroBooNE is, what the low energy excess is, and how one searches for the latter in the former will be described in detail.\\

To begin, Chapter \ref{sec:theory} will provide some introductory theoretical background about neutrinos both within and beyond the Standard Model of physics. Following that, Chapter \ref{sec:detector} will describe the MicroBooNE detector, located at the Fermi National Accelerator Laboratory in Batavia, Il. This detector employs a liquid argon time projection chamber (LArTPC), a relatively new technology for neutrino detection, especially at the size of MicroBooNE (with order of meters drift distance). Next, Chapter \ref{sec:beam} will describe the Booster Neutrino Beam (BNB), the beam of neutrinos produced at Fermilab by colliding primary protons with a beryllium target and focusing the outgoing charged secondaries. The relatively large uncertainties associated with the neutrino flux will be introduced, which are particularly important for the analyses described in this thesis. Chapter \ref{sec:LEEhistory} goes on to describe in detail the excess of electron-like events seen in primarily $\nu_\mu$ beams, first by the LSND collaboration and then again by the MiniBooNE collaboration. Ultimately the MiniBooNE collaboration was unable to resolve whether the excess is of electron-like events, or photon-like events due to limitations of the detector technology. For this reason, the MicroBooNE experiment was proposed to measure the same neutrino beam at a similar location to MiniBooNE, but with the different LArTPC detector technology that has photon/electron discrimination capabilities.\\

Chapter \ref{sec:LEEsensitivity} describes a simulation-based analysis done to estimate the sensitivity of the MicroBooNE detector to measure a MiniBooNE-like excess, with some assumptions about that excess. This is an important analysis to identify which sources of backgrounds are most relevant to this search in order to step closer to an eventual search for the excess in MicroBooNE data. As will be described, the dominant background to this search is the intrinsic $\nu_e$ contamination in the beam, about half of which come from $K^+$ production at the BNB proton target. There is a relatively large flux uncertainty associated with with this $K^+$ production, which is the subject of Chapter \ref{sec:kaon}.\\

Chapter \ref{sec:kaon} presents the first steps toward a $K^+$ production at the BNB primary proton target measurement in MicroBooNE. This analysis selects and analyses high energy $\nu_\mu$ interactions in MicroBooNE, which provide a pure sample of $\nu_\mu$ from $K^+$ decay. The \textit{in situ} measurement of these interactions is used to constrain the aforementioned important intrinsic $\nu_e$ from $K^+$ background for the low energy excess search.\\

In order to make the kaon production measurement, calculating the energy of several-GeV muons from $\nu_\mu$ interactions in MicroBooNE is a necessary step. Given the muon kinematics in liquid argon, these particles travel on average many meters and almost always exit the active detector volume. Appendix \ref{sec:MCS} presents a publication (whose first author is the author of this thesis) describing in detail the multiple Couloumb scattering based method used to estimate the energy of muons which exit a LArTPC. The publication describes important discovery made about the underlying phenomenological formula which past LArTPC neutrino experiments have neglected: the formula needs to be re-tuned for use specifically in liquid argon.\\

The thesis concludes with a summary of the results of the three analyses described, along with the future prospects for those analyses and for the MicroBooNE experiment in general.

% The introduction will be brief, and will introduce the reader to the document. It will outline the flow of the thesis: start with a description of neutrinos and some basic neutrino theory, then describe the MicroBooNE detector, then describe the BNB, then dive in to the Low Energy Excess section which contains description of LSND, description of MiniBooNE, description of the analysis in MicroBooNE, the sensitivity results in MicrobooNE. That section ends with describing how the flux systematic on intrinsuc nues from kaon decay (an important background in the LEE search) can be improved with an in-situ measurement of numus from kaons. Then comes the numu-from-kaon section, which references the importance of multiple Coulomb scattering and I will copy/paste the JINST publication in there. Then there will be a short conclusions/summary section, followed by any appendices which are needed.\\

% ``You have 3 topics in your thesis: LE, K’s and MCS. But they are very well related. SO a sentence or two in the Intro linking them would be great. Something along these lines:\\

% In LE the intrinsic nue’s are an important irreducible background, of which a significant component is from kaons produced at the proton target decaying to nue’s. A measure of these kaons can be obtained from the studu of the high energy numuCC energy distribution. The momentim of the high energy muons produced in these events can only be measured by the multiple scatterin technique as they exit the cryostat precluding their measurement through range.'' --Leslie