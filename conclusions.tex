Following some introductory material including a description of neutrinos and neutrino oscillations, the liquid argon time projection technique and specifically the MicroBooNE detector at the Fermi National Accelerator Lab, and the results and implications of the LSND and MiniBooNE experiments, three subsequent MicroBooNE analyses have been described. All three of these analyses are closely interconnected, and are ultimately geared towards searching for and understanding the MiniBooNE measured low energy excess of electromagnetic events in the MicroBooNE detector.\\

First, a detailed sensitivity analysis was described. It involved simulating beam neutrino events in the MicroBooNE detector, and using reconstruction algorithms to select them. These algorithms included leveraging 3D energy deposition ($dE/dx$) information to separate electrons from photons, a capability which MiniBooNE did not possess. Additionally, MiniBooNE low energy excess public data releases were used to simulate what the excess would look like in MicroBooNE, assuming it was induced by an excess of beam $\nu_e$ interactions. Ultimately, a sensitivity was estimated for the MicroBooNE detector to observe such a signal. The dominant background in this search was (irreducible) intrinsic $\nu_e$ from both pion and kaon decay in the beam-line, and the flux uncertainty associated with the production of these particles is large.\\

The first steps toward the analysis to measure the $K^+$ production in the beam-line by selecting high energy $\nu_\mu$ events was presented. This analysis constrains the significant portion of intrinsic $\nu_e$ backgrounds which come from kaon decay in the beam-line. This important measurement was initially done by the SciBooNE collaboration, but its reproduction in MicroBooNE is relevant because the MicroBooNE detector has a different neutrino target material (argon) than SciBooNE (polystyrene), and the $K^+$ production measurement can be done \textit{in situ} in the same detector searching for the low energy excess.\\

An important part of the $K^+$ analysis involves measuring the energy of muons created in $\nu_\mu$ interactions which exit the detector. While measuring the energy of fully contained muons is straightforward, the only method to measure that of exiting tracks is by means of multiple Coulomb scattering. A detailed investigation into how the MCS algorithm employed by MicroBooNE works and quantification of its performance both in data and simulation was given.\\

The future prospects for MicroBooNE are bright. As of the time this thesis was written, the experiment has collected roughly 80\% of the nominal protons-on-target agreed to be delivered by Fermilab. While the collaboration has struggled to develop algorithms to automatically reconstruct the data, an effort is being made to analyze data with deep learning convolutional neural networks~\cite{UBCNNsource}. This may ultimately be MicroBooNE's path towards measuring the MiniBooNE low energy excess. Additionally, the MicroBooNE detector will serve as one of three detectors in the Short Baseline Neutrino experiment ~\cite{SBNproposal} which promises to deliver the most sensitive search to date for sterile neutrinos at the eV mass-scale. This is an exciting time for precision neutrino measurements!