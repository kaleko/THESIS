\section{Low Energy Excess Overview and Motivation}
\subsection{Introduction}
The purpose of this chapter is to describe the MicroBooNE analysis centered around measuring the observed low energy excess as seen by its predicessor experiment, MiniBooNE. This chapter begins with a historical motivation for this analysis by describing the LSND experiment which first observed an excess of $\overline{\nu_e}$ in a $\overline{\nu_\mu}$. Then, a detailed description of the MiniBooNE experiment (detector and analysis) is provided, which observed an unexplained excess of electron-like events in the neutrino energy region between 200 to 475 MeV.\\

With historical context in perspective, this chapter then discusses an analysis conducted which determines the sensitivity of the previously described MicroBooNE detector to measure the same signal as MiniBooNE, assuming that signal originated from an excess of beam-induced $\nu_e$ interactions. This discussion covers the signal modeling which comes from MiniBooNE published data releases, the event selection and background mitigation techniques in MicroBooNE, and ultimately the expected sensitivity to measure such a signal.

\subsection{LSND Observation}
%Brief discussion of the LSND experiment and how they saw an excess of antinue in the antinumu beam with L/E that didn't fit in with any measured mixing angles and delta m2 in 3 neutrino model.

In 2001, the Liquid Scintillator Neutrino Detector (LSND) collaboration published an observation of excess events consistent with $\overline{\nu_e}$ interactions above the expected background in an $\overline{\nu_\mu}$ beam at the Los Alamos Neutron Science Center \cite{LSNDPaper}. The $\frac{L}{E}$ ($\approx$ $\frac{30 m}{40 MeV}$) for this excess disagreed with previous measurements of neutrino mixing angles and $\Delta m^2$ values in the three neutrino model. The LSND excess corresponded to a $\Delta m^2$ of approximately 1 $eV^2$, significantly higher than previously measured values of $\Delta m_{12}^2$ and $\Delta m_{23}^2$. One explanation for this drastically different $\Delta m^2$ value is the existance of potential additional ``sterile'' neutrino states, which must not interact weakly given the Z- boson decay width constrains the number of weakly interacting neutrino states to three.\\
To test the LSND result, the MiniBooNE experiment was designed. It would similarly measure $\nu_e$ interactions in a primarily $\nu_\mu$ beam, with a similar $\frac{L}{E}$ ($\approx$ $\frac{500m}{700 MeV}$).



\subsection{The MiniBooNE Experiment}

\subsubsection{The MiniBooNE Detector and Monte Carlo Simulation}

%Description of the detector, a schematic figure, description of cherenkov rings, the location of the detector in the BNB. Discussion of how they use NUANCE, mention they use identical beam simulation and reweighting (reweighting should be covered in the beam chapter).


The MiniBooNE detector \cite{MBDetectorPaper} consists of a spherical tank located 541 meters downstream of the BNB neutrino production target, with diameter of 12.2 meters filled with 818 tons of mineral oil underneath at least 3 meters of earth overburden as shown in Figure \ref{MB_detector_fig}. As seen in the figure, there exists an opaque barrier separating a veto region and a signal region. The walls of the signal region are painted black to reduce re-scattering of light and are instrumented with 1280 8-inch photomultiplier tubes (PMTs), most of which were reused from the LSND experiment. In the signal region, the photocathode coverage is 11.8\%. This region is intended to identify beam neutrino interactions happening within it; the PMTs face radially inwards. The veto region is 35 cm thick and is instrumented with 240 PMTs which face tangent to the detector radius and its purpose is to reject backgrounds coming from outside of the detector (cosmics, which have a rate at the detector location of about 10 kHz). The inner surface of the veto region is coated in a reflective paint. The efficiency for rejecting cosmic ray muons with the outer veto region was measured to be 99.99\%.\\

\begin{figure}[ht!]
\centering
	\includegraphics[width=0.9\textwidth]{Figures/MB_detectorpaper_fig.png} \\
\caption{\textit{The MiniBooNE detector enclosure (left) and a cut-away drawing (right) of the detector showing the distribution of PMT's in the signal and veto regions.}}\label{MB_detector_fig}
\end{figure}

The detection method of the MiniBooNE experiment is based primarily on Cherenkov light. The mineral oil within the signal region acts as the neutrino target material. The majority of final state particles exiting neutrino interactions at neutrino energies from the BNB are produced above Cherenkov threshold. These particles produce Cherenkov light which is detected by the PMTs lining the signal region of the detector. Reconstructing the pattern this light projects onto the walls of the signal region allows for some particle identification abilities.\\

Thie mineral oil used in MiniBooNE is Marcol 7 Light Mineral Oil ($CH_2$). While the oil has lower density than water (a common material choice for Cherenkov detectors) and therefore smaller probability of a neutrino interacting within the detector, the other benefits outweigh this downside. The oil has good light transmission throughout the wavelength range of 320 nm to 600 nm, relatively large refractive index (1.47, greater than water at 1.33), and its long extinction length of greater than 20 m for 420 nm light. This extinction length allows for loss of no more than 25\% of light generated by a neutrino interaction at the center of the detector. Additionally, the Cherenkov threshold is lower than that of water for the final state particles of interest (electrons, pions, muons, protons), allowing for the measurement of lower energy particles.\\

The PMTs have a wavelength dependent efficiency with a peak at 390 nm, with half that efficiency at 315 and 490 nm. They are operated at +2000 V which results in a gain of approximately $1.6 \times 10^7$. They have an intrinsic time resolution on the order of 1 ns, which is the dominant contribution to the final time resolution in the final PMT data after readout through data acquisition (DAQ) electronics. The DAQ reads out a pmt when the charge signal corresponds to more than 0.1 photoelectrons (PE). The deadtime between successive PMT readouts is on the order of 250 ns after the first readout began. PMT charge and timing information is stored in intervals of 200 $\mu s$ following any trigger signal. The trigger signal of interest for this analysis is the beam trigger which is induced by the BNB accelerator clock such that the DAQ begins readout 5 $\mu s$ before each beam spill.\\

The detector is calibrated in situ primarily from cosmic ray muons. Within the detector are six optically sealed scintillator cubes. Incominb muons are tagged with a muon hodoscope above the detector with angular resolution better than 2 degrees. Those muons that stop within one of the scintillation cubes have well defined energy (in the 100-800 MeV range), as the stopping power of muons in mineral oil is well known, and therefore provide an energy calibration source. Additionally, the outgoing electrons from stopping muon decay (Michel electrons) have a well known endpoint of around 50 MeV and therefore serve as an electron energy calibration source at that energy. Through-going muons are also used as an energy calibration source at higher energies (above 1 GeV). Additional calibration sources include a laser system built into the detector, and reconstructed $\pi^0$ masses (with peak energy at 135 MeV).

\subsubsection{MiniBooNE Event Selection and Observed Excess}
% Brief description of their event selection cuts, mention that the energy definition they use is CCQE and they're looking specifically for CCQE topologies, description of what CCQE means. Figure of their results indicating the excess in neutrino mode. 

Different final state particles exiting a neutrino interaction in the MiniBooNE signal volume will create different patterns of Cherenkov light read out by the PMTs. Figure \ref{georgia_cherenkov_cartoon_fig} \cite{GeorgiaThesis} shows how these patterns differ for different common kinds of final state particles (muons, electrons/photons, and neutral pion decays). A muon track produces a crisp, filled-in ring of Cherenkov light, while an electron or photo produces a more fuzzy, hollow ring. A neutral pion decay will result in two photons. By reconstructing these patterns in the PMT data read out from a triggered event in MiniBooNE, the type and energy of the interacting neutrino can be determined. With this kind of detection technique it is important to note that a single photon signal is indistinguishable to that of a single electron signal, an important ingredient to the ultimate ambiguity of the observed low energy excess in MiniBooNE.\\

\begin{figure}[ht!]
\centering
	\includegraphics[width=0.9\textwidth]{Figures/georgia_cherenkov_cartoon.png} \\
\caption{\textit{A schematic of the pattern Cherenkov light from different particles would make projected onto the inner walls of the MiniBooNE detector. Top is a muon track (a filled-in ring), middle is an electron (a fuzzy ring), bottom is a photon that pair-produces and creates two fuzzy rings.}}\label{georgia_cherenkov_cartoon_fig}
\end{figure}

The topology of interest in the MiniBooNE oscillation search is that of charged-current quasi-elastic (CCQE) interactions, shown in Figure \ref{georgia_ccqe_feynman_fig}. This interaction channel is the dominant one in the neutrino energy range of the BNB, around 1 GeV $E_\nu$. In a $\nu_l$ CCQE interaction (where $l$ is the neutrino flavor), a lepton of flavor $l$ is produced, along with a proton. The single outgoing lepton is the characteristic event signature for which MiniBooNE searches.\\


\begin{figure}[ht!]
\centering
	\includegraphics[width=0.9\textwidth]{Figures/georgia_ccqe_feynman.png} \\
\caption{\textit{Feynman diagrams of the charged-current quasi-elastic (CCQE) interaction channel for $\nu_e$, $\nu_\mu$, $\overline{\nu_\mu}$, and $\overline{\nu_e}$ (clockwise from the top left). $\nu_e$ CCQE is the signal channel for the MiniBooNE oscillation analysis.}}\label{georgia_ccqe_feynman_fig}
\end{figure}


In a $\nu_e$ appearace search, the intrinsic $\nu_e$ in the beam are an irreducible background. Aside from this, the most dominant background in the low energy region is from $\pi^0$ misidentifications. In such an event, a $\pi^0$ is created in the neutrino interaction and its subsequent immediate decay into two photons mimics a the $\nu_e$ CCQE signature (either one photon escapes, or rings overlap).\\

In order to reconstruct events, MiniBooNE uses a maximum likelihood fitting algorithm leveraging properties of charged particle tracks inferred from measured charges and times on the PMTs. The likelihoods to different event hypothesis are used to classify each event as a signal $\nu_e$ CCQE event, or as a background process like $\nu_\mu$ CCQE and NC $\pi^0$ production. Note that MiniBooNE cannot differentiate between a $\mu^+$ and a $\mu^-$, or $e^+$ and $e^-$ so discrimination between neutrino and antineutrino on an event-by-event basis is impossible.\\

Assuming CCQE kinematics, the incident neutrino energy can be reconstructed with knowledge of the outgoing lepton energy ($E_l$) and scattering angle ($\theta_l$). In MiniBooNE specifically, the struck nucleon is assumed to be at rest.

\begin{equation}\label{MB_CCQE_formula}
E_\nu^{CCQE} = \frac{2m_nE_l+m_p^2-m_n^2-m_l^2}{2(m_n-E_l+\cos\theta_l\sqrt{E_l^2-m_l^2})}
\end{equation}
















































\subsubsection{Proposed Low Energy Excess Sources}
The LEE could either be electron like or photon like, MiniBooNE couldn't tell. Mention of theories like sterile neutrinos though no models seem to fit very well (3+1 or 3+2 with possible CP violation), single photon background misestimations or unexpected backgrounds, neutrino decay, lorentz violation etc etc. 

\subsection{MicroBooNE In The Context of the Low Energy Excess}
Discussion of how as a LArTPC MicroBooNE has electron/photon separation. Showing the scaled signal plot from the TDR. Mention that this scaled signal is oversimplified so that's why this thesis describes a more rigorous sensitivity study with simulation in MicroBooNE.



\section{Monte Carlo Simulation}

\subsection{Simulated Background Samples}
Mention that the input samples to this study are simulated BNB interactions inside of the entire cryostat, simulated cosmic samples, simulated intrinsic nue samples. Mention the stats are low because simulation is computationally intensive. Mention that GENIE is the generator. Mention the same flux reweighting is done as in MiniBooNE (and the same flux simulation is used as described in the beam chapter).

\subsection{Reconstruction}
Currently the automated reconstruction in MicroBooNE is not adequate to do any sort of sensitivity study (specifically shower reconstruction). Discussion that the input objects into this sensitivity are showers and tracks, description of what they are and how they differ. 
\subsubsection{``Perfect Reconstruction''}
Description of what {\sc MCTracks} and {\sc MCShowers} are, how they're made, etc. Make it clear that they are the input to the event reconstruction algorithms. Also make it clear that this entire sensitivity study is done only with these objects; no real automated reconstruction is covered (which is why no results on data are shown).



\section{Event Reconstruction}
This section describes how we take reconstructed objects (tracks and showers) and identify nue interactions. Mention that the analysis framework has many algorithms that each serve a specific purpose. Briefly mention the algorithms by name and what they're intended to do. Include the flowchart figure from the APS technote listing all the algorithms and the order in which they are executed. The following sections will describe in more detail the important ones. 

\subsection{Electron/Photon Separation Algorithm}
This is the algorithm (AlgoEMPart) that uses the reconstructed dE/dx and conversion distance to form a likelihood that a shower is electron like or photon like. It's one of the most important and should be described in detail. Include some figures showing performance of this algorithm.

\subsection{Nue Selection Algorithm}
This is the algorithm (AlgoSingleE) designed to look for the signal topology of CCInclusive... one electron plus anything else from the vertex. Mention the efficiency, etc.

\subsection{Energy Reconstruction}
Description of the energy definition. Mention that it is different than the CCQE energy definition MiniBooNE used. Include a figure of Ereco vs Etrue for the intrisic nue sample.



\section{Backgrounds}

\subsection{Background Topologies}
Description of the various background topologies (michel electrons from cosmics and numu, pi0s where one shower escapes, pi0s where one shower enters in dirt backgrounds, etc.)

\subsection{Background Normalization}
Mention the beam induced backgrounds are normalized to POT, the cosmic backgrounds come from the open cosmic sample and are normalized to beam gate open time (which is assuming perfect flash matching).

\subsection{Analysis Cuts and Results}
Description of the final cuts that are placed on the backgrounds (for example requiring the electron deposits more than 60 MeV of energy to mitigate michels, also requiring no pions in order to best mimic the MiniBooNE cuts). Showing the stacked backgrounds.



\section{MiniBooNE Low Energy Excess Signal Modeling In MicroBooNE}
Here is where I describe how I come up with my scaled signal shape and normalization. Simulated sample are intrinsic nues, since I'm assuming the excess is coming from beam nues. Shape comes from MiniBooNE published data (excess events evis distribution and uz distribution along with the CCQE formula). Normalization comes from size of MiniBooNE excess with respect to size of MiniBooNE intrinsic nue background in a specific region (excess should be larger than the intrinsic nue background in the low CCQE-energy region... this was Bill's recent suggestion). This section will probably be several pages long.

\subsection{Sensitivity Results}
Here I show the stacked background with the scaled signal on top, I describe how I compute a sensitivity (statistical errors only for now... I could do a back-of-the-envelope estimate of systematic errors which are dominated by flux). 
\subsubsection{Results with Realistic Shower Reconstruction Efficiency}
Here I mention that the ``perfect reconstruction'' efficiency is 100\% but that isn't quite realistic. I'll state we assumed 80\%, and ICARUS quoted something similar. I'll describe how we emulate the non-perfect efficiency and how it isn't as simple as multiplying everything by 0.8 (for example we will get increased pi0 backgrounds when we fail to reconstruct just one of the showers).
\subsubsection{Next Steps}
Here I talk about how what's next is automated shower reconstruction being incorporated. Another important ingredient in the whole LEE analysis is constraining the nue backgrounds. The intrinsic nues which come from kaon decay can be constrained by studying numu interactions that come from kaon decay. This section will flow into the next chapter of the thesis: kaon production studies.

