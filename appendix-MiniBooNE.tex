\chapter{Additional MiniBooNE Details}\label{MINIBOONE_APPENDIX}

This appendix provides some additional details about the MiniBooNE detector, and about the experiment's event selection techniques in their low energy excess analysis.

\section{MiniBooNE Detector}\label{MINIBOONE_APPENDIX_DETECTOR}
\subsection{Detector Material}
The mineral oil used in MiniBooNE is Marcol 7 Light Mineral Oil ($CH_2$). While the oil has lower density than water (a common material choice for Cherenkov detectors) and therefore smaller probability of a neutrino interacting within the detector, the other benefits outweigh this downside. The oil has good light transmission throughout the wavelength range of 320 nm to 600 nm, relatively large refractive index (1.47, greater than water at 1.33), and its long extinction length of greater than 20 m for 420 nm light. This extinction length allows for loss of no more than 25\% of light generated by a neutrino interaction at the center of the detector. Additionally, the Cherenkov threshold is lower than that of water for the final state particles of interest (electrons, pions, muons, protons), allowing for the measurement of lower energy particles.\\
\subsection{PMT Specifications}
The PMTs have a wavelength dependent efficiency with a peak at 390 nm, with half that efficiency at 315 and 490 nm. They are operated at +2000 V which results in a gain of approximately $1.6 \times 10^7$. They have an intrinsic time resolution on the order of 1 ns, which is the dominant contribution to the final time resolution in the final PMT data after readout through data acquisition (DAQ) electronics. The DAQ reads out a PMT when the charge signal corresponds to more than 0.1 photoelectrons (PE). The dead time between successive PMT readouts is on the order of 250 ns after the first readout began. PMT charge and timing information is stored in intervals of 200 $\mu s$ following any trigger signal. The trigger signal of interest for this analysis is the beam trigger which is induced by the BNB accelerator clock such that the DAQ begins readout 5 $\mu s$ before each beam spill.\\
\subsection{Calibrations}
The detector is calibrated \textit{in situ} primarily from cosmic ray muons. Within the detector are six optically sealed scintillator cubes. Incoming muons are tagged with a muon hodoscope above the detector with angular resolution better than 2 degrees. Those muons that stop within one of the scintillation cubes have well defined energy (in the 100-800 MeV range), as the stopping power of muons in mineral oil is well known, and therefore provide an energy calibration source. Additionally, the outgoing electrons from stopping muon decay (Michel electrons) have a well known endpoint of around 50 MeV and therefore serve as an electron energy calibration source at that energy. Through-going muons are also used as an energy calibration source at higher energies (above 1 GeV). Additional calibration sources include a laser system built into the detector, and reconstructed $\pi^0$ masses (with peak energy at 135 MeV).
